\chapter{Závěr}
Výsledkem naši práce je aplikace pro přístroje s operačním systémem Android, vyvinutá pro občanské sdružení IuRe. 
\paragraph{}
Přestože jsme s vývojovým prostředím pro tuto platformu neměli žádné zkušenosti, bohatá dokumentace tématu a propracované prostředí IDE Eclipse umož\v{n}uje programování i nováčkům v oboru. Přesto je stále mnoho prostoru pro zlepšování a proto, že se vývoji aplikace budeme věnovat i nadále, není verze odevzdávaná v rámci předmětu finální. V budoucnosti bude předělána grafická stránka aplikace.  Z důvodů snížení datové náročnosti bude implementována metoda, která stáhne umístění kamer pouze v bližším okolí. Dalším nezbytností je zabezpečení aplikace proti zneužitelnosti, aplikováním časového omezení vstupu nových kamer z přístroje, který bude možné dále identifikovat unikátním k\'{o}dem. Nakonec bude pod technickým zázemím organizace spuš\v{t}en server pro komunikaci.
\paragraph{}
Při práci na projektu jsme osobně ocenili praktický dopad aplikace, technickou náročnost a nové zkušenosti s vývojem mobilní aplikace. Zadání bylo splněno s dvěma výjimkami. Jednou z nich je komunikace mezi aplikací a databází pomocí php skriptů. Místo nich jsme použili \texttt{Apache Tomcat} a javovské servlety, protože se jedná o mnohem lepší řešení. Druhou výjimkou je nahrávání kamer z aplikace rovnou do databáze \texttt{OpenStreetMap}. To není z technického hlediska úplně možné a proto jsme se rozhodli tento problém obejít ukládáním kamer do databáze na našem serveru, do které jsou naimportovány i data z \texttt{OpenStreetMap}. 